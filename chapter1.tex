\chapter{Une rencontre inattendue}

\paragraph{}
Les ténèbres avaient envahi depuis plusieurs heures la grande capitale du
Kremnon: Fresmost. Cette cité glaçait le sang par son uniformité géométrique
et son absence de courbes. La rigueur du climat avait contraint les habitants
à limiter autant que possible le nombre et la surface des ouvertures. Le seul
indice de la présence de vie résidait dans les quelques lumières vacillantes
qui patrouillaient dans les rues afin d'assurer la sécurité de la population.
La démarche peu assurée des gardes illustrait leurs craintes, ils savaient
que bien qu'ils soient toujours par pair, ils n'étaient pas assez nombreux
pour faire face à une embuscade.

\paragraph{}
Dans leurs échoppes, les marchands ne dormaient que d'un oeil, craignant pour
leurs biens plus que pour leurs vies. Les nuages qui planaient au-dessus de la
ville rendaient cette nuit encore plus dangereuses, sans la lumière de la lune
pour illuminer les rues, nul ne pouvait savoir qui s'y cachait. Cette
obscurité offrait un refuge de choix pour les voleurs.

\paragraph{}
Un craquement dans sa boutique suffit à sortir l'armurier de son sommeil
léger. Il alluma une bougie et descendit les escaliers aussi vite que sa panse
rebondie le lui permettait. Arrivé en bas, il vit sa porte se refermer sur une
ombre. Paniqué, il émit un cri de détresse: ``Au voleur!'' En ouvrant sa
porte, il se retrouva nez à nez avec deux membres de la garde qui lui
demandèrent d'une voix sèche et entièrement dépourvue de sympathie:
``Où est-il parti?''

\paragraph{}
C'est à ce moment précis qu'il perçut un lointain sourire en provenance de
l'autre côté de la rue. L'ombre qui venait de s'échapper de chez lui le
regardait fixement, exhibant le butin dont il venait de s'emparer. Tremblant
de peur, il pointa d'un doigt accusateur l'homme qui le narguait. Le temps que
les gardes se retourne, il s'était mis à courir dans la direction opposée. Les
soldats s'élancèrent soudainement, bien décidés à ne pas perdre le malfaiteur
de vue.

\paragraph{}
Alourdis par le poids de leur armure ils n'étaient pas capable de soutenir
l'allure de la silhouette qui les distançait peu à peu. Comprenant qu'ils ne
pourraient pas l'attraper sans aide, le plus petit des gardes s'empara d'une
corne dans sa besace et souffla de toute ses forces.

\paragraph{}
Deux virages plus tard, d'autres lumières se dirigèrent vers eux, forçant le
voleur à dévier de sa route. Sans un mot, ils continuèrent leur poursuite,
convaincus de sa réussite. Ils connaissaient le quartier, leur proie venait
de se diriger dans un cul de sac et même s'ils venaient de la perdre de vue,
ce n'était plus qu'une question de temps avant qu'ils ne le rattrapent. L'un
d'eux hurla alors sa victoire: ``Nul n'échappe à la milice de Fresmost!'', la
seule réponse qui leur parvint fut un rictus porté par le vent.

\paragraph{}
Lorsqu'ils aperçurent les maisons qui annonçaient la fin de la poursuite, ils
entendirent soudain un bruit métallique, suivis d'étranges paroles. Horrifié,
l'un des gardes s'arrêta soudainement en s'exclamant: ``De la magie, c'est un
magicien!''. Les lumières des torches éclairèrent alors la paroi d'un mur,
illuminant la silhouette qui grimpait le long d'une corde, déjà à quelques pas
du sol.

\paragraph{}
Sans hésitation, le plus gradé de la troupe sortit son arc en ordonnant aux
autres gardes de faire de même: ``Tuez-le! Il ne nous échappera pas cette
fois-ci!''. Le voleur s'était déjà hissé sur le toit lorsque les gardes
décochèrent leurs premières flèches. Trop loin, pour assurer des tirs précis,
ils manquèrent leur cible. Conscient du peu de chance qu'ils avaient de le
toucher, un des gardes s'approcha du mur afin de suivre sa proie sur les
toits, mais la corde avait disparu.

\contextswitch

\paragraph{}
Assis sur la crête d'un toit, Eldorn contemplait son butin de la soirée. Il
n'avait jamais vu d'aussi belle paire de dagues que celles dont il venait de
s'emparer. Le métal dont étaient composées les lames lui était totalement
inconnu, et il semblait ne refléter aucune lumière. Lorsqu'il avait sorti les
armes de leur fourreau, il avait voulu profiter d'un creux dans les nuages
pour les observer plus attentivement, mais la faible clarté des étoiles lui
avait alors semblées entièrement absorbée par l'étrange matière qu'il était
incapable d'identifier. Il ne parvenait pas à déterminer si cette
particularité était le fruit d'un quelconque enchantement, mais il avait une
certitude, la valeur de ces dagues était inestimable.

\paragraph{}
Une bourrasque le sortit soudainement de sa contemplation en le faisant
frissonner. Ce n'avait jamais été la peur de tomber qui l'avait fait quitter
les toits, mais plutôt le froid. Souhaitant alors un peu de chaleur, il
se remit en marche.

\paragraph{}%Magie
Quelques toits plus loin, il vérifia soigneusement qu'aucune patrouille ne
pouvait le voir et il accrocha sa corde à une cheminée. Il descendit le
long d'une façade délabrée, pour finalement atterir devant une porte qui
semblait en piteux état. Il murmura un mot étrange et la corde qu'il tenait
entre ses mains se retourna soudainement. Il s'approcha de la porte et frappa
trois coups rapides puis deux coups longs et attendit.

\indent- Qui vient donc à cette heure tardive? L'interrogea une voix féminine
 emplie d'inquiétude.\newline
\indent- C'est moi.

\paragraph{}
La porte s'ouvrit immédiatement, découvrant un visage souriant aux cheveux
blonds, perché au sommet d'un corps d'une taille impressionnante. Son
interlocutrice la dépassait d'environ un demi-pied. Celle-ci murmura alors
d'une voix pressée: ``Entre vite Eldorn, il est déjà tard, je ne suis pas
sensée t'ouvrir!''
