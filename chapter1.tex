\chapter{Une rencontre inattendue}

\paragraph{}
Les ténèbres avaient envahi depuis plusieurs heures la grande capitale du
Kremnon: Fresmost. Cette cité glaçait le sang par son uniformité géométrique
et son absence de courbes. La rigueur du climat avait contraint les habitants
à limiter autant que possible le nombre et la surface des ouvertures. Le seul
indice de la présence de vie résidait dans les quelques lumières vacillantes
qui patrouillaient dans les rues afin d'assurer la sécurité de la population.
La démarche peu assurée des gardes illustrait leurs craintes, ils savaient
que bien qu'ils soient toujours par pair, ils n'étaient pas assez nombreux
pour faire face à une embuscade.

\paragraph{}
Dans leurs échoppes, les marchands ne dormaient que d'un oeil, craignant pour
leurs biens plus que pour leurs vies. Les nuages qui planaient au-dessus de la
ville rendaient cette nuit encore plus dangereuses, sans la lumière de la lune
pour illuminer les rues, nul ne pouvait savoir qui s'y cachait. Cette
obscurité offrait un refuge de choix pour les voleurs.

\paragraph{}
Un craquement dans sa boutique suffit à sortir l'armurier de son sommeil
léger. Il alluma une bougie et descendit les escaliers aussi vite que sa panse
rebondie le lui permettait. Arrivé en bas, il vit sa porte se refermer sur une
ombre. Paniqué, il émit un cri de détresse: ``Au voleur!'' En ouvrant sa
porte, il se retrouva nez à nez avec deux membres de la garde qui lui
demandèrent d'une voix sèche et entièrement dépourvue de sympathie:
``Où est-il parti?''

\paragraph{}
C'est à ce moment précis qu'il perçut un lointain sourire en provenance de
l'autre côté de la rue. L'ombre qui venait de s'échapper de chez lui le
regardait fixement, exhibant le butin dont il venait de s'emparer. Tremblant
de peur, il pointa d'un doigt accusateur l'homme qui le narguait. Le temps que
les gardes se retourne, il s'était mis à courir dans la direction opposée. Les
soldats s'élancèrent soudainement, bien décidés à ne pas perdre le malfaiteur
de vue.

\paragraph{}
Alourdis par le poids de leur armure ils n'étaient pas capable de soutenir
l'allure de la silhouette qui les distançait peu à peu. Comprenant qu'ils ne
pourraient pas l'attraper sans aide, le plus petit des gardes s'empara d'une
corne dans sa besace et souffla de toute ses forces.

\paragraph{}
Deux virages plus tard, d'autres lumières se dirigèrent vers eux, forçant le
voleur à dévier de sa route. Sans un mot, ils continuèrent leur poursuite,
convaincus de sa réussite. Ils connaissaient le quartier, leur proie venait
de se diriger dans un cul de sac et même s'ils venaient de la perdre de vue,
ce n'était plus qu'une question de temps avant qu'ils ne le rattrapent. L'un
d'eux hurla alors sa victoire: ``Nul n'échappe à la milice de Fresmost!'', la
seule réponse qui leur parvint fut un rictus porté par le vent.

\paragraph{}
Lorsqu'ils aperçurent les maisons qui annonçaient la fin de la poursuite, ils
entendirent soudain un bruit métallique, suivis d'étranges paroles. Horrifié,
l'un des gardes s'arrêta soudainement en s'exclamant: ``De la magie, c'est un
magicien!''. Les lumières des torches éclairèrent alors la paroi d'un mur,
illuminant la silhouette qui grimpait le long d'une corde, déjà à quelques pas
du sol.

\paragraph{}
Sans hésitation, le plus gradé de la troupe sortit son arc en ordonnant aux
autres gardes de faire de même: ``Tuez-le! Il ne nous échappera pas cette
fois ci!''. Le voleur s'était déjà hissé sur le toit lorsque les gardes
décochèrent leurs premières flèches. Trop loin, pour assurer des tirs précis,
ils manquèrent leur cible. Conscient du peu de chance qu'ils avaient de le
toucher, un des gardes s'approcha du mur afin de suivre sa proie sur les
toits, mais la corde avait disparu.

\contextswitch

\paragraph{}
Assis sur la crête d'un toit, Eldorn contemplait son butin de la soirée. Il
n'avait jamais vu d'aussi belle paire de dagues que celles dont il venait de
s'emparer. Le métal dont étaient composées les lames lui était totalement
inconnu, et il semblait ne refléter aucune lumière. Lorsqu'il avait sorti les
armes de leur fourreau, il avait voulu profiter d'un creux dans les nuages
pour les observer plus attentivement, mais la faible clarté des étoiles lui
avait alors semblées entièrement absorbée par l'étrange matière qu'il était
incapable d'identifier. Il ne parvenait pas à déterminer si cette
particularité était le fruit d'un quelconque enchantement, mais il avait une
certitude, la valeur de ces dagues était inestimable.

\paragraph{}
Une bourrasque le sortit soudainement de sa contemplation en le faisant
frissonner. Ce n'avait jamais été la peur de tomber qui l'avait fait quitter
les toits, mais plutôt le froid. Souhaitant alors un peu de chaleur, il
se remit en marche.

\paragraph{}%Magie
Quelques toits plus loin, il vérifia soigneusement qu'aucune patrouille ne
pouvait le voir et il accrocha sa corde à une cheminée. Il descendit le
long d'une façade délabrée, pour finalement atterrir devant une porte qui
semblait en piteux état. Il murmura un mot étrange et la corde qu'il tenait
entre ses mains se retourna soudainement. Il s'approcha de la porte et frappa
trois coups rapides puis deux coups longs et attendit.

\indent- Qui vient donc à cette heure tardive? L'interrogea une voix féminine
 emplie d'inquiétude.\newline
\indent- C'est moi.

\paragraph{}
La porte s'ouvrit immédiatement, découvrant un visage souriant aux cheveux
blonds, perché au sommet d'un corps d'une taille impressionnante. Son
interlocutrice la dépassait d'environ un demi pied. Celle-ci murmura alors
d'une voix pressée: ``Entre vite Eldorn, il est déjà tard, je ne suis pas
sensée t'ouvrir!''

\paragraph{}
Celui-ci ne se fit pas attendre, il referma rapidement la porte derrière lui
et fit coulisser le verrou. À peine la porte s'était-elle refermée qu'une
autre porte s'ouvrit, découvrant une grande salle où un feu crépitait. Le
petit espace dans lequel il se trouvait servait uniquement de sas, afin
d'éviter que de la lumière soit aperçue de l'extérieur lorsque la porte
d'apparence miteuse s'ouvrait. Le contraste entre l'aspect usé de la façade du
bâtiment et son intérieur chaleureux n'était pas commun, il servait
principalement à éviter d'attirer l'attention, les habitants du lieu n'ayant
aucun intérêt à ce qu'on soupçonne leur présence.

\paragraph{}
Eldorn balaya la salle commune du regard, personne d'autre n'était présent.
Une main se glissa doucement dans ses cheveux, cherchant à masquer le ton
inquisiteur de la voix qui lui demanda: ``Qu'as-tu donc fait pour rentrer si
tard ce soir?''

\paragraph{}
Les yeux bruns du jeune voleur pétillèrent alors de malice, il répondit
nonchalamment: ``Aujourd'hui, je me baladais dans l'allée des marchands en
délestant quelques bourses, lorsque soudain j'ai aperçu deux splendeurs qui
ont retenu mon regard. Elles m'ont tant fasciné que j'en ai conclu qu'il
fallait absolument que je leur rende une petite visite durant la nuit. C'est
donc ce que j'ai fait...''

\paragraph{}
Au fur et à mesure de son explication, il avait senti la main qui caressait sa
tête se crisper. Il se retourna alors pour observer la réaction qu'il avait
suscitée, satisfait d'y voir un mélange de tristesse et de colère, il termina
alors son explication: ``Je les ai même ramenées, tu veux les voir?''

\paragraph{}
Il retira son manteau et dégaina les deux dagues dont il s'était emparé plus
tôt dans la soirée. La lumière ne se reflétait toujours pas dans les lames,
mais des symboles étranges étaient à présent visibles. Il sourit alors, fier
de montrer son acquisition à son aînée. Sentant qu'elle posait sur lui un
regard admiratif, une envie lui traversa soudainement l'esprit et d'un geste
fluide, il fit pivoter ses dagues et les rangea dans leurs fourreaux. Sans
perdre plus de temps, il passa un bras dans son dos et l'embrassa.

\paragraph{}
Loin de lui résister, elle répondit très favorablement à son approche en
l'entourant de ses bras. Il se laissa alors guider sur une voie qu'elle
connaissait sûrement mieux que lui. Il sentit une main délicate
se glisser sous sa tunique, effleurant le bas de son dos. En un instant, la
douce caresse se transforma en poigne de fer et il se sentit propulsée contre
le corps de son aînée. Il fut alors plongé dans un univers de sensations
inconnues, perdant tout contrôle sur les événements et jouissant de
la chaleur d'un corps féminin contre le sien. Mais au coeur de cette ivresse,
un sentiment plus fort que les autres perçait. Il était incapable de
déterminer sa provenance, mais malgré ses efforts, il ne pouvait l'ignorer,
quelque chose le perturbait et lui nouait l'estomac. Il aurait voulu
s'abandonner entièrement et la laisser l'emmener dans ce joyeux tourbillon
d'émotions intenses, mais il ne pouvait pas.

\paragraph{}
Ne souhaitant pas révéler ses faiblesses, il interrompit leur baiser et se
contenta de murmurer: ``Il faut tout de même que j'aille faire mon rapport,
sinon je risque de passer un sale quart d'heure!'' L'étreinte s'assouplit
soudainement et il revint progressivement à la réalité. Sans avoir besoin de
s'y attarder, il put lire sur le visage de Melyane un mélange de joie,
d'inquiétude et de frustration. Se sentant coupable de la quitter ainsi, il
lui dit d'un ton rassurant: ``Je reviens dès que possible.''

\paragraph{}
Eldorn grimpa les escaliers d'un pas rapide, son sentiment de malaise avait
disparu et il se sentait fier de la situation qu'il avait acquise. Il y a un
peu plus d'un an, il était arrivé à Fresmost après ce qui lui avait semblé une
interminable errance. Il avait passé un mois à mendier en essuyant les regards
méprisants des passants plus fortunés. Lorsqu'il avait trop faim, il volait
quelques fruits et du pain afin de survivre. Il avait aussi essayé d'être
honnête, cherchant des petits travaux à effectuer, mais lorsqu'il s'était
rendu compte qu'il gagnait beaucoup moins que les intermédiaires alors qu'il
prenait tous les risques, il avait vite abandonné. C'est ces mauvaises
expériences qui l'avait fait se méfier de la guilde des voleurs lorsqu'elle
l'avait approché pour la première fois, il craignait qu'elle ne lui reverse
qu'une part infime des gains. Mais, depuis qu'il les avait rejoint, il n'avait
plus eu faim et il avait toujours une couche chaude ou dormir. De plus, il
avait eu l'impression d'intégrer une grande famille et ses mentors étaient
rapidement devenus plus proche de lui que ne l'avaient jamais été ses parents.
Entouré d'individus avides de liberté, il se sentait dans son élément.

\paragraph{}
Arrivé à l'étage, il se retrouva face à un lurnain de moins de cinq pieds de
haut. Il reconnut immédiatement son mentor et s'exclama avec fierté: ``Borlin!
Il faut que je te montre ce que j'ai volé ce soir!'' Sans lui laisser le
temps de répondre, Eldorn dévoila son butin devant le regard admiratif de son
mentor qui lui répondit: ``Je crois bien que l'élève a finit par dépasser le
maître!'' Borlin poussa un bref soupir et reprit d'un air réprobateur: ``Si
seulement ta fougue ne t'entraînait pas à prendre des risques insensés, si tu
réfléchissais avant de t'exposer au danger, tu pourrais entreprendre de grands
projets et changer le monde. Tu pourrais peut-être même mettre un terme à la
dynastie Dragoryan! Pourtant, au milieu de tes prouesses, tu fais de très
mauvais choix... J'ai compris ce que tu viens de débuter avec Melyane et
c'était une des pire choses que tu pouvais faire. Tu t'es engagé sur un chemin
inconnu sans te soucier de s'il te convenait où non. Maintenant, crois-moi, tu
ne peux plus faire machine arrière. Je souhaites seulement que tu conserves
ton insouciance et que tu ne la déçoives pas. Car si tu ne lui donnes pas ce
qu'elle veut, elle te le fera payer.''

\paragraph{}
Eldorn avait de la peine à réaliser ce qu'il venait d'entendre. Ce n'était pas
la première fois que son mentor lui tenait un discours aussi sombre, mais il
n'arrivait toujours pas à s'y habituer. On lui avait toujours décrit les
lurnains comme de bons vivants, avides d'histoires drôles, insouciants et
toujours prêts à rire. Il se souvint de ce que lui avait dit son mentor
lorsqu'il l'avait interrogé à ce sujet: ``Il faut toujours te méfier des
généralités, il n'y a rien de plus dangereux. J'ai vu bien des choses dans ma
vie, tu sais que j'ai beaucoup voyagé, mais pour toutes les règles que j'ai
entendues, j'ai toujours trouvé des exceptions. Même le bien et le mal auquel
tant de monde s'accrochent, ça n'existe pas. Il y a simplement des différences
de point de vue, de l'incompréhension et aussi parfois des oppositions
idéologiques trop forte pour permettre une coexistence pacifique. La plus
grande et la plus absurde des sources de conflits que je connaisse étant
l'appartenance à des groupes distincts. Si chacun se pensait plus en tant
qu'individu qu'en tant que membre d'une classe, d'une race ou d'une idéologie,
alors on éviterait de verser beaucoup de sang et on pourrait mener une
existence pacifique. Bien sûr, quand je dis tout ça, tu dois te demander
pourquoi est-ce que j'ai rejoins la guilde des voleurs si je pense ainsi. Les
raisons sont nombreuses, j'ai mes convictions, mes plaisirs et mon dégoût de
l'obéissance. Je ne souhaite pas vivre seul, et il est toujours bon d'avoir
des amis pour couvrir ses arrières. Ensemble, on peut lutter pour essayer
d'apporter du changement, on peut essayer de réveiller ce peuple somnolent et
on peut faire peur aux seigneurs. Mais je garde toujours à l'esprit le plus
important, si le comportement de la guilde me déçoit ou si je suis
agréablement surpris par le comportement d'un de ses adversaires, j'agirai en
tant qu'individu et non en tant que membre de la guilde.''

\paragraph{}
Ces paroles avaient alors intrigué l'adolescent, sans qu'il réussisse à en
saisir le sens. Lorsque Borlin parlait, Eldorn savait que c'était avec
sagesse, mais cela ne l'aidait pas à mieux comprendre. Il trouvait le
comportement de ses parents mauvais et la guilde bonne envers lui, après tout
ce qu'elle lui avait apporté, il se sentait éternellement reconnaissant et
était prêt à tout faire pour la guilde.

\paragraph{}
Il fut arraché à ses pensées par la voix bienveillante de son mentor qui
tentait de le rassurer: ``Tu as toute ta vie pour réfléchir à ce que je t'ai
dit, on a plus qu'à veiller à ce qu'elle ne s'écourte pas trop! Va voir
Fremnar maintenant, il a une tâche à te confier.'' Borlin descendit alors les
escaliers, comme chaque nuit, il allait s'endormir auprès du feu, se
languissant du climat plus tempéré de sa Flormyle natale.

\paragraph{}
Eldorn traversa lentement le couloir, arrivé au bout du long et étroit
passage, il ouvrit une porte sur sa droite et dit: ``Vous vouliez me voir?''

\paragraph{}
À quelques pas de lui, Fremnar le toisait d'un regard réprobateur du haut de
ses six pieds et demi. D'une voix grave et caverneuse en accord avec son
épaisse carrure, il le sermonna: ``Eldorn, tu sais bien que j'ai horreur que
l'on désobéisse à mes ordres...'' Le jeune voleur ravala difficilement sa
salive, il essayait de se repasser en mémoire les ordres qu'il avait pu
recevoir. N'en trouvant pas, il répondit d'une voix hésitante: ``Fremnar, tu
sais bien que je respecte tes ordres, car je sais que tu n'en donnes qu'en cas
d'absolue nécessité. Je ne crois pas t'avoir désobéit à ce jour.''

\paragraph{}
Un sourire amusé se profila le long du visage de son interlocuteur qui lui dit
d'un ton paternel: ``Je me disais justement que j'avais bien fait de ne pas
chercher à contrôler tes heures de rentrée, j'aurais passé mon temps à devoir
te punir. Maintenant, dis-moi pourquoi tu rentres si tard, il ne reste que
quelques heures avant l'aube.'' Eldorn éclata d'un rire nerveux, ce n'était
pas la première fois qu'il se laissait duper par Fremnar.

\paragraph{}
Après avoir raconté ses exploits et s'être fait féliciter, le jeune voleur
devint muet et attentif lorsqu'il sentit que le sujet devenait plus sérieux.
D'une voix de chef expérimenté, Fremnar reprit calmement: ``Tu m'as posé une
question tout à l'heure et je vais maintenant y répondre. Je voulais
effectivement te parler car demain, Borlin, Melyane et moi allons partir à
l'aube pour effectuer une mission qui risque de durer quelques jours.
J'aimerais que tu nous accompagnes pour nous épauler, mais tu as bien entendu
le droit de refuser si tu ne te sens pas prêt.''

\paragraph{}
Eldorn peinait à contenir sa joie, tout semblait lui réussir ce soir, le vol
des dagues s'était très bien passé, puis il y avait eu ce baiser avec Melyane
et voilà que Fremnar lui proposait de participer à une mission qui devait être
de haute importance. Obnubilé par le contenu de la mission, il s'exclama:
``Bien sûr que j'accepte! J'attendais ce moment avec impatience! Quel est le
but de la mission et quel va être mon rôle dans celle-ci?''

\paragraph{}
Fremnar sourit, il appréciait la nature enthousiaste du jeune homme. Il prit
une profonde respiration et commença ses explications: ``C'est un contrat
aussi inhabituel qu'important que nous allons accomplir demain. Si nous
l'honorons, nous renflouerons nos caisses, nous causerons quelques inquiétudes
aux dirigeants de Termion et nous sauverons une innocente. En revanche, si
nous échouons, il ne vaut mieux pas penser à ce qui nous arrivera. Le plus
étrange dans cette affaire est que nous allons travailler pour des nobliaux,
ça ne m'est arrivé que très rarement, mais je me ferai un plaisir d'exécuter
ce contrat!''

\paragraph{}
Eldorn fut surpris par cette déclaration, la plupart du temps, la guilde
travaillait pour son propre compte. Il arrivait fréquemment qu'un marchand
fasse appel à leurs services pour s'emparer de pièces rares qui ne se
vendaient pas, mais en un an il n'avait jamais entendu parler de contrat avec
des nobles. Ceux-ci faisaient partie de leurs ennemis, et les aider ne lui
serait pas venu à l'esprit.

\paragraph{}
Fremnar reprit ses explications avec sérieux: ``Nous allons devoir récupérer
une jeune noble prénommée Tendrya et la protéger. Elle partira de Fresmost
demain pour s'en aller vers la demeure de sa mère. Elle sera accompagnée de
gardes qui selon nos informations sont corrompus. Nous devrons intervenir
rapidement après qu'ils soient sorti de la ville, car son escorte risque de
vouloir se débarrasser d'elle aussi vite que possible. Notre mission sera de la
garder en vie en un lieu sûr afin qu'elle échappe aux assassins de la famille
Dragoryan. Pour ce qui est du plan, tu seras chargé de la ramener à notre
repère dans les bois de la hantise, en veillant bien à ne pas être suivi.
Borlin et moi allons faire diversion à l'orée de la forêt, dès que l'attention
de l'escorte sera détournée, tu en profiteras pour t'emparer de la fille.
Melyane nous attendra tous au repère afin de s'assurer que la voie soit
libre. Est-ce que tu as bien compris?''

\paragraph{}
Pendant qu'Eldorn acquiesçait, Fremnar en profita pour reprendre une gorgée de
bière avant de reprendre: ``Il faut que tu sois bien conscient de trois choses
pendant la durée de l'opération. Tout d'abord, la fille ne s'attend pas du
tout à notre opération, elle sera donc certainement effrayée et elle se
sentira inévitablement menacée. Elle fait partie de la noblesse, le plus dur
sera donc certainement de supporter ses jérémiades. Ensuite, il ne faut pas
que les gardes se doutent que nous l'enlevons pour la protéger. Finalement, il
ne faut pas qu'elle soit blessée, je n'ai aucune envie de devoir rendre un
cadavre ou une fille recouverte de cicatrice à sa mère. Alors n'hésite pas à
prendre des risques pour qu'elle s'en sorte indemne et si tu arrives à gagner
sa confiance, je pense que ton travail n'en sera que plus facile. Je sais très
bien que c'est un peu exigeant comme première mission à mes côtés, mais Borlin
et moi pensons que tu es le plus apte à remplir ce rôle. Échapper à des
poursuivants est presque une seconde nature chez toi.''

\paragraph{}
Une pointe d'orgueil perça alors dans l'esprit d'Eldorn, il se rendait compte
de l'estime et de la confiance que lui portait certains des membres les plus
influents de la guilde et il avait une seule envie: s'en montrer digne. Malgré
son manque d'expérience, on lui confiait une tâche extrêmement importante pour
sa première vraie mission. Dans la guilde des voleurs, il se sentait
réellement apprécié et valorisé pour ce qu'il était et non pour ce qu'il
aurait pu être.

\paragraph{}
La voix grave de Fremnar reprit d'un ton moins assuré, il cherchait ses mots
tout en parlant, conscient d'aborder un problème épineux: ``Eldorn, on
reconnaît tous ici que tu es un excellent élément et personne ne conteste ton
appartenance à la guilde. On a tous placé beaucoup d'espoirs en toi et on se
doute bien que tu nous dépasses déjà dans certains domaines. Mais le plus
important c'est que nous t'aimons avec ton caractère et ta spontanéité,
certains comme un fils et d'autres comme un frère. Mais tu dois faire
attention, certains sentiments ont tendance à créer plus de problèmes qu'à en
régler. D'autant plus que tu n'as pas la maturité nécessaire pour prendre
conscience de la différence entre tes désirs et ceux des autres. Tu ignores
encore beaucoup des choses de la vie, même si l'on t'a appris beaucoup de ce
que nous savons, il y a des choses que la vie t'apprendra. Je te demande de
faire attention aux chemins que tu empruntes, n'oublie jamais que les plus
grands dangers se dissimulent parfois très bien. Je ne voudrais pas te voir
nous quitter, mais je préfère ça à te survivre parce que tu as été trop
imprudent. Réfléchis bien avant de t'engager sur de nouveaux chemin, sois sûr
de connaître leur destination.''

\paragraph{}
Bien qu'il ait été habitué à entendre Borlin communiquer de manière indirecte,
Eldorn était surpris de voir Fremnar aborder un sujet de manière aussi
détournée. Exaspéré, il l'interrompit: ``Je comprends rien à ce que tu
racontes! Explique-toi plus clairement si tu veux que tes avertissements
soient utiles.''

\paragraph{}
Le géant reprit sa respiration, malgré son courage sans faille lorsqu'il
s'agissait de combattre, parler de sentiments le dérangeait toujours, d'autant
plus lorsqu'il ne s'agissait pas des siens. Il reprit d'une voix chancelante:
``Tu dois faire attention à Melyane, elle s'intéresse beaucoup à toi. Au
moindre signe d'approbation de ta part, elle aura l'impression d'avoir des
droits sur toi. Elle est bien plus puissante que tu ne le penses, et si tu la
déçois, elle risque de te le faire payer. Tu as deux possibilités, soit tu
restes distant et tu la laisses se lasser de toi, soit tu t'engages à rester
avec elle, mais dans ce cas là, prépare-toi à ce qu'elle prenne le contrôle de
ta vie. Elle deviendra très rapidement jalouse de toute autre femme qui pourra
te parler.''

\paragraph{}
Eldorn avait écouté les propos de son chef sans y prêter trop d'attention, il
lui semblait impossible qu'une femme aussi douce que Melyane puisse être aussi
possessive, elle qui l'avait tiré des griffes de la rue. Il avait
effectivement remarqué qu'elle n'était pas insensible à ses avances quelques
instants plus tôt, mais il ne s'imaginait pas un seul instant qu'elle n'ait
d'espérances concrètes avec lui, après tout, elle était plus âgée. Il
remercia Fremnar pour ses conseils en le rassurant sur le fait qu'il les
prendrait en compte, puis il s'éclipsa afin de dormir quelques heures avant
que leur mission ne commence.

\paragraph{}
Arrivé près de la chambre où il dormait habituellement, il fut surpris d'y
voir la lueur d'une bougie. Il n'en avait nullement besoin, ses origines lui
conférant une excellente vision nocturne. Il fut étonné de trouver Melyane
allongée dans la couchette. Comme elle semblait dormir, il éteint la bougie et
s'apprêta à sortir pour se mettre en quête d'une chambre vide. Avant qu'il
n'atteigne la porte, un doux murmure sensuel lui parvint aux oreilles: ``Vient
dormir avec moi Eldorn, il y a encore de la place.''

\paragraph{}
Les avertissements qu'il avait reçu lui semblaient à présent prendre forme. Il
n'avait pas imaginé un seul instant qu'elle soit si enthousiaste. Comme elle
le sentait hésitant, Melyane ajouta d'une voix coupable: ``Je suis désolée, je
pensais que tu aurais apprécié que je t'attende, je voulais seulement
t'exprimer à quel point j'étais fière de toi.'' Elle fit mine de se relever,
mais Eldorn lui fit signe de rester, l'air triste qu'elle avait affiché ainsi
que sa manière de le flatter l'avait fait changé d'avis. Il ne comprenait pas
comment ses mentors pouvaient lui demander de se méfier d'elle alors qu'elle
était si fragile: ``Ne t'inquiète pas, c'est seulement que comme demain
j'aurai d'importantes responsabilités, je voulais me reposer un peu pour être
en forme.'' Sans attendre une seule seconde de plus, elle lui répondit avec
douceur: ``Je comprends, tu dois être stressé. Je vais te faire un petit
massage et après je retournerai dans ma chambre, tu dormiras mieux ainsi.
Enlève seulement ta tunique et allonge-toi sur le ventre.'' Elle lui sourit
alors d'un air si innocent qu'il n'eut pas le coeur de lui refuser. Il se
délesta de ses dagues et de sa tunique au pied du lit et s'allongea.

\paragraph{}
Il sentit tout d'abord des mains douces lui passer le long du dos, dénouant
ses muscles fatigués. Il ferma les yeux et il était sur le point de s'endormir
lorsqu'il sentit un souffle d'air chaud monter le long de sa nuque. L'instant
d'après, elle l'embrassait à pleine bouche, avant qu'il ne comprenne ce qui
était en train de se passer, il sentit une main se glisser sous son torse,
descendre jusqu'à son nombril puis poursuivre sa route.

\paragraph{}
D'un mouvement sec, Eldorn s'éloigna de Melyane et laissa échapper la vérité:
``Non! Pas maintenant! Je ne suis pas encore prêt... j'ai besoin de plus de
temps.'' Melyane pouffa alors de rire et s'exclama: ``Allons... tu as seize
ans, beau comme tu l'es tu ne vas pas me faire croire qu'il n'y a eu personne
avant moi!'' Elle sentit immédiatement la gêne du jeune homme et murmura alors
d'une voix honteuse: ``Je suis désolée, je ne pensais vraiment pas que c'était
vrai.'' Elle reprit alors d'une voix plus enthousiasmée: ``Tu verras, c'est un
peu étrange au début, mais tu seras à l'aise avant de t'en rendre compte.
Laisse-moi seulement te montrer le chemin...'' Comme elle faisait mine de
l'attirer contre lui, il réagit plus fort qu'il ne l'aurait voulu: ``Non, j'ai
dit non, c'est tout! Il faut que tu apprennes à être patiente.''

\paragraph{}
Melyane se releva soudainement et s'écarta de lui, sans élever sa voix au-delà
du murmure, elle parvenait à transmettre l'intensité de sa fureur: ``Patiente?
Tu veux que je sois patiente? Tu devrais peut-être l'être un peu moins. Il y a
des lunes que j'attends que tu me remarques, je crois que tu dois être le
dernier de la guilde à t'être rendu compte des attentions que je te porte.
Pourquoi crois-tu que j'attendais dans la salle commune ce soir? Pourquoi
crois-tu que je reste éveillée à t'attendre chaque soir? Tu n'as pas envie que
je sois patiente, tu veux simplement que je fasse ce que tu veux et quand tu
le veux. Mais je ne suis pas un simple jouet, j'ai mes envies et mes besoins
aussi. Il serait peut-être temps que tu apprennes à écouter un peu les autres
et que tu cesses d'être aussi égocentrique. Tu es rentré en montrant tes
dagues, en exhibant ton succès, mais as-tu seulement demandé à qui que ce soit
ce qu'il avait fait pendant sa soirée? T'es-tu inquiété de la souffrance de
ceux qui ne s'en sortent pas aussi bien que toi? Non, tu veux être le centre
de l'attention, tu veux que tout le monde t'admire... si seulement tu pouvais
me témoigner un centième de l'attention que je te témoigne, tu ferais de moi
la plus heureuse des femmes.''

\paragraph{}
Sans un mot, elle se dirigea vers la sortie de la pièce en sanglotant. Cloué
par le discours assassin, Eldorn faillit ne pas réagir à temps. Il venait
seulement de comprendre à quel point elle l'aimait, et après tout ce qu'elle
avait fait pour elle, il ne pouvait pas la laisser repartir dans cet état. Il
la rattrapa et déposa une main sur son épaule. Incapable de supporter plus
longtemps la détresse de la jeune femme, il lui dit ce qu'elle avait envie
d'entendre: ``Je suis vraiment désolé pour mon comportement, je t'aime et
j'aimerais pouvoir te l'exprimer de tout mon corps, mais ce soir je suis
simplement trop obsédé par la mission de demain. Dès qu'elle sera terminée, je
te promets que je serai entièrement à toi.'' Cela suffit à faire cesser les
pleurs de Melyane, mais il sentit que ses craintes ne s'étaient pas totalement
apaisées, d'une voix tenue elle lui dit: ``Je crois qu'il vaut mieux que je
dorme ailleurs ce soir. Repose-toi bien et ne t'inquiète pas trop pour
demain, tu seras à la hauteur.'' Elle l'embrassa avec douceur et il la laissa
quitter la pièce.

\paragraph{}
Au lieu de s'endormir paisiblement comme il l'avait souhaité, il ne pouvait
s'empêcher de réfléchir à ce qui venait de se passer, à ce qu'elle venait de
lui dire, à l'engagement qu'il venait de prendre. Elle l'aimait réellement, il
n'osait pas l'espérer quelques heures plus tôt, ne se sentant pas digne d'une
telle attention, et voilà qu'à présent cette réalité le terrifiait plus
qu'elle ne le réjouissait. Après le discours qu'elle lui avait tenu, il ne
se sentait pas plus digne d'elle, il était obsédé par ce qu'elle lui avait
dit. L'avait-elle accusé ainsi simplement dans le but de le blesser ou
était-il vraiment devenu si imbu de lui-même qu'il ne faisait plus attention
aux autres?

\contextswitch

\paragraph{}
Eldorn sortit de son sommeil troublé et entendit du bruit en provenance de la
salle commune, il en déduit que l'aube devait s'être levée. Il enfila
rapidement ses vêtements, s'assura d'avoir bien prit son équipement et fini
par se diriger vers la salle commune.

\paragraph{}
Il y trouva Fremnar et Borlin qui discutaient ensemble, ils avaient déjà
revêtu leurs armures de cuir et leurs épées les attendaient, appuyées contre
la table. Eldorn leur demanda d'une voix encore un peu endormie: ``Alors, on
part dans combien de temps?'' La voix grave et caverneuse de Fremnar lui
répondit d'un ton assuré: ``Tu peux vite manger quelque chose le temps que
j'aille réveiller Melyane. On partira sans elle mais il faut tout de même
qu'elle se prépare car il faudra qu'elle arrive à notre repaire avant vous. Tu
auras certainement besoin d'un peu de repos en arrivant là-bas et il faudra
que quelqu'un veille à ce que Tendrya ne tente pas de s'échapper.''

\paragraph{}
Eldorn dévora rapidement une miche de pain, puis il enfila une épaisse veste
en cuir afin de se protéger du froid mais aussi d'offrir une légère protection
en cas de combat. À peine avait-il finit que Fremnar était déjà de retour,
leur faisant signe de sortir.

\paragraph{}
Dehors, le soleil venait tout juste de se lever, sans chercher à réfléchir
plus profondément à leur itinéraire, Eldorn suivait ses mentors d'un pas
décidé. Ses appréhensions disparaissaient peu à peu, laissant place à sa
détermination de se montrer à la hauteur de leurs espérances.

\contextswitch

\paragraph{}
Couché sur sa branche depuis un moment, Eldorn fixait l'orée de la forêt sans
ciller. Peu lui importait de rester immobile durant des heures, il avait
appris à contracter et à détendre ses muscles sans mouvements afin qu'ils ne
s'engourdissent pas pendant les longs temps d'attente lors d'embuscade. Rien
ne comptait plus à ses yeux que le nuage de poussière qui venait d'apparaître.
Méthodiquement, il comptait le nombre de cavaliers, il y avait six gardes en
armure ainsi qu'une silhouette encapuchonnée, son objectif probablement.

\paragraph{}
D'un bref signe de main, il indiqua la teneur de la troupe à Borlin et Fremnar
qui sortirent alors des buissons pour se tenir au travers de la route, épées
dégainées. L'emplacement n'avait pas été laissé au hasard, le chemin se
faisait particulièrement étroit à cet endroit précis et il n'était donc pas
possible pour la petite troupe de cavalier de penser simplement éviter les
deux hommes armés qui bloquaient le passage. Lorsque celle-ci arriva à leur
hauteur, les chevaux s'arrêtèrent et les gardes descendirent tous
simultanément. Seule la silhouette encapuchonnée était restée sur sa monture,
elle semblait jeter des regards inquiets autour d'elle.

\paragraph{}
Il ne put s'empêcher de sourire en observant l'incompréhension qui régnait
chez les gardes lorsque Fremnar clama un tonitruant: ``La bourse ou la vie!''
Le capitaine des gardes s'exclama alors d'un ton méprisant: ``Vous n'avez pas
choisi les bonnes proies, c'est une erreur que vous ne ferez pas deux fois.''

\paragraph{}
L'instant d'après, le fracas des armes se mit à parler. D'un bond, Eldorn se
propulsa derrière son objectif et se réceptionna silencieusement. En quelques
secondes, il s'était installé sur le cheval, juste derrière la silhouette
encapuchonnée qui poussa alors un cri de frayeur. À quelques pas d'eux, le
combat s'arrêta soudainement, tandis que le cheval faisait demi-tour et se
lançait au galop. Comprenant soudain la situation, le capitaine hurla: ``En
selle! Elle ne doit pas s'échapper!'' Puis il se précipita vers sa propre
monture, s'empara de son arc qui était resté accroché à sa selle, encocha une
flèche, prit une grande inspiration, visa d'un oeil expert et laissa la corde
vibrer. Malgré la distance, il ne manqua pas entièrement sa cible, il perçut
un mouvement brusque de la part du cavalier qui s'enfuyait et il sut alors
qu'il l'avait touché. Ne souhaitant prendre aucun risque, il s'élança lui
aussi à leur poursuite.

\paragraph{}
Fremnar et Borlin profitèrent de la confusion qui régnait parmi les gardes
pour en tuer trois avant qu'ils ne regagnent leurs montures. Ils se jetèrent
alors un regard complice, satisfaits de leurs travail. Puis, laissant les
cadavres sur place, ils s'engouffrèrent dans la forêt, marchant paisiblement
le long d'un sentier à peine tracé.

\paragraph{}
Bien que la plaie d'Eldorn ne soit pas profonde, elle le brûlait, il avait
de la peine à tenir les rênes et à maintenir Tendrya en place. Lorsqu'elle
avait entendu la flèche siffler près de ses oreilles, elle s'était encore
plus affolée et il avait du raffermir sa prise pour éviter qu'elle ne tombe.
Exaspéré, il lui dit d'une voix sèche: ``Cette flèche ne m'était pas
destinée, c'est toi qui était visée. Si tu veux survivre, fais-moi confiance
et suis mes ordres.'' Soudainement conscient de la dureté dont il faisait
preuve, il essaya de la calmer d'une voix plus douce: ``Tu es Tendrya, c'est
bien ça?''

\paragraph{}
Sa voix harmonieuse apaisa un peu la jeune femme qui lui demanda alors d'un
air étonné: ``Comment connaissez-vous mon nom et pour quelle raison avez-vous
entrepris de me sauver?'' Incapable de réfléchir correctement tout en tenant
une discussion, Eldorn l'abrégea en essayant de ne pas être trop sec: ``Je
t'expliquerais une fois que nous serons sortis d'affaire, d'accord?'' Elle
lui signifia alors son accord d'un léger hochement de tête.

\paragraph{}
À présent plus libre de ses mouvements, le jeune voleur jeta un bref coup
d'oeil en arrière. Au vu de la distance à laquelle le suivait le nuage de
poussière, il comprit qu'il n'avait pas pris autant d'avance qu'il l'aurait
souhaité et qu'ils étaient en train de se faire rattraper. À la recherche
d'une solution, il parcourut alors les alentours du regard. C'est en voyant
le branchage dense en-dessus de lui qu'il eut finalement une idée. Il murmura
alors à Tendrya: ``Retourne-toi si tu peux et accroche-toi bien à moi après
le prochain virage.''

\paragraph{}%Magie!
Eldorn attrapa alors la longue corde qu'il avait à sa ceinture, puis la lança
en l'air juste après un virage en épingle. Il murmura une brève incantation
et la corde se noua d'elle-même autour d'une branche suffisamment épaisse
pour supporter leur poids. L'instant d'après, ils s'élevèrent dans les airs,
laissant leur monture poursuivre son galop. Le jeune acrobate se réceptionna
avec légèreté sur une branche, à proximité d'un tronc. Ne souhaitant pas
risquer la chute de Tendrya, il l'aggripa immédiatement et l'aida à se
stabiliser sur la branche. Il vérifia qu'ils n'avaient pas fait tomber trop
de feuilles et murmura: ``Surtout ne bouge plus et ne fais plus de bruit,
ils vont sûrement mettre un moment à réaliser que nous avons disparu, ça
nous laissera le temps de nous écarter des grandes routes sans laisser trop
de traces.''

\paragraph{}
Il guetta alors leurs poursuivants et vit passer un groupe de deux soldats
puis quelques instants après un autre, seul. Ne voyant plus aucune poussière
se soulever sur la route, il jugea que le danger principal était écarté. Il
fit signe à Tendrya de le suivre et commença à faire le tour de l'arbre
lentement en lui montrant bien quelles branches utiliser. Malgré ses
indications, elle semblait effrayée et son déplacement était peu assuré. Il
ne fallut pas longtemps à Eldorn pour comprendre que s'il souhaitait qu'elle
arrive vivante à destination, il valait mieux retourner sur la terre ferme.
Il accrocha donc une corde à une branche facile d'accès et se laissa glisser
délicatement le long de celle-ci. Debout sur l'herbe, il fit signe à Tendrya
de l'imiter. Comme elle semblait effrayée par les quatre pas qui la séparait
du sol, il l'encouragea en lui expliquant comment faire. Peu à peu elle
gagna en assurance, Eldorn se surprit à penser qu'elle avait de belles jambes
lorsqu'elle atteint finalement le sol.

\paragraph{}
Leur court périple aérien avait ôté la capuche de Tendrya, c'est en posant
les yeux sur son visage qu'Eldorn surprit soudain un détail intriguant:
les pointes de ses cheveux étaient d'un rouge éclatant